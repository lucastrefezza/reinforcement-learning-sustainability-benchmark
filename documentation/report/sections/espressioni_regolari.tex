\chapter{Espressioni Regolari}
\label{ch:espressioni_regolari}
Le espressioni regolari sono strettamente correlate agli automi a stati finiti, sia deterministici che non deterministici, sono infatti in grado di descrivere la stessa classe di linguaggi: quella dei linguaggi regolari. A differenza delle descrizioni basate su una tipologia di macchina, si tratta di una descrizione algebrica~\cite{hopcroft:alc}, utile in molte applicazioni pratiche, e spesso più succinta dell'NFA o del DFA equivalente.

Le espressioni regolari sono spesso utilizzate come linguaggio di input per sistemi che trattano stringhe, tra i tanti possiamo indubbiamente ricordare i comandi di ricerca nei sistemi UNIX, in cui l'utente utilizza le espressioni regolari per descrivere i \textit{pattern} che intende ricercare all'interno dei file; successivamente un componente software apposito le converte in un DFA o in un NFA, poi simulato sui file da esaminare. Un'altra delle applicazioni più significative è quella dei generatori di analizzatori lessicali, il componente del compilatore che divide il programma sorgente in unità dette \textit{token}, la cui forma è descritta tramite espressioni regolari appunto.

Essendo le espressioni regolari utilizzate per denotare linguaggi, la loro notazione, ed in particolare quella dei tre operatori principali che le compongono, riprende tre operazioni fondamentali sui linguaggi. Queste operazioni sono l'unione, la concatenazione e la chiusura, anche detta star o chiusura di Kleene. Non descriveremo le tre operazioni nell'ambito dei linguaggi formali in quanto date per assunte, ma in seguito alla definizione di un'espressione regolare, forniremo quella dei linguaggi da esse definite.

\paragraph{Definizione formale di espressione regolare}
\begin{definizione}
	Sia $\Sigma$ un alfabeto, un'espressione regolare su $\Sigma$ è definita ricorsivamente nel modo seguente:\\
	\textbf{Passo Base:}
	\begin{enumerate}
		\item $\forall a \in \Sigma$,  $a$ è una espressione regolare;
		\item $\epsilon$ è una espressione regolare;
		\item $\emptyset$ è una espressione regolare.
	\end{enumerate}
	\textbf{Passo Ricorsivo:} se $R_1$ ed $R_2$ sono espressioni regolari, allora:
	\begin{enumerate}
		\item $(R_1)$ è una espressione regolare;
		\item $R_1 \cup R_2$ (o, equivalentemente, $R_1 + R_2$) è una espressione regolare;
		\item $R_1 \cdot R_2$ (o, equivalentemente e più comunemente, $R_1R_2$) è una espressione regolare;
		\item $R_1^*$ è una espressione regolare.
	\end{enumerate}
\end{definizione}

L'ordine della precedenza degli operatori è il seguente: star, concatenazione, unione. Per forzare un ordine diverso diventa necessario l'utilizzo delle parentesi, che sono altrimenti facoltative, salvo per l'applicazione di un'operatore a porzioni più ampie di un'espressione regolare piuttosto che ai soli simboli più vicini, ad esempio $ab^*$ ed $(ab)^*$ sono due espressioni regolari diverse che denotano linguaggi diversi: nel primo caso l'operatore star è applicato alla sola $b$, nel secondo ad $ab$.

Alla definizione di base appena fornita diversi autori aggiungono altre definizioni per pura comodità, a tutti gli effetti del syntactic sugar: sia $R$ un'espressione regolare, definiamo $R^+ = RR^*$, otteniamo quindi che $R^*~=~R^+ \cup \epsilon$. Definiamo inoltre la concatenazione di $k$ copie di $R$ come $R^k, \text{ con } k \in \Nz$ (e di conseguenza $R^0 = \epsilon$).

Completata la definizione di una espressione regolare, passiamo alla definizione del linguaggio da questa denotata. Una espressione regolare $R$ è appunto un'espressione, per indicare il linguaggio da essa denotata si utilizza la notazione $L(R)$.

\paragraph{Definizione ricorsiva dei linguaggi rappresentati dalle espressioni regolari su un alfabeto $\Sigma$}
\begin{definizione}
	Sia $\Sigma$ un alfabeto, il linguaggio di un'espressione regolare $R$ su $\Sigma$ è indicato con $L(R)$ ed è definito ricorsivamente nel modo seguente:\\
	\textbf{Passo Base:} siano $\epsilon$ e $\emptyset$ espressioni regolari
	\begin{enumerate}
		\item $\forall a \in \Sigma$, $a$ espressione regolare,  $L(a) = \{a\}$;
		\item $L(\epsilon) = \{\epsilon\}$;
		\item $L(\emptyset) = \emptyset$.
	\end{enumerate}
	\textbf{Passo Ricorsivo:} se $R_1$ ed $R_2$ sono espressioni regolari, allora:
	\begin{enumerate}
		\item $L((R_1)) = L(R_1)$;
		\item $L(R_1 \cup R_2) = L(R_1 + R_2) = L(R_1) \cup L(R_2)$;
		\item $L(R_1 \cdot R_2) = L(R_1R_2) = L(R_1)L(R_2)$;
		\item $L(R_1^*) = L(R_1)^*$.
	\end{enumerate}
\end{definizione}