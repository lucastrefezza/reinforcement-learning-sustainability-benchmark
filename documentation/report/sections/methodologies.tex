\section{Methodological Steps to Conduct to Address the Goals}
\label{sec:methodologies}

The methodology used follows from the basic idea of this benchmark: to execute all the algorithms for the same number of environment interactions, so that we can compare the score they achieve and the energy consumption of each one of them. Additionally, a good comparison would be to take the score obtained by the lowest performer in this initial trial and re-train all the algorithms until they reach that score. This would allow us to compare how much time and energy each algorithm requires to achieve the same performance level. Unfortunately, time and resources constraints make retraining all algorithms unfeasible, so we will approximate this second comparison by using the returns from the logging of the training during the first trial. This logging includes the \verb*|global_step|, indicating the environment interaction we are at, and the \verb*|episodic_return|, which is the return of the episode (i.e., the score on which to compare), as well as all performance and power consumption data up to that point. By analyzing these logs, we will estimate how much time and energy each algorithm would take to reach the score obtained by the lowest performer in the initial trial. 

The following sections outline the several key steps involved in the methodology adopted for this project.

\subsection{Algorithms Selection}
\label{subsec:algorithm_selection}
%In our benchmark we will consider, regardind the first category, the DQN, which constitutes the first example of success of deep reinforcement learning (so that we have a sort of baseline), and RAINBOW, a method that involves a lot of the tweaks and improvement made to the original DQN. In addition to these two, we will test various of the single tweaks to assess their individual contribution to energy consumption and performance, and more advanced methods like SPR (Self-Predictive Representations, introduced in the fifth work cited).
%
%Regarding policy gradient and actor critic methods, we will start with a basic one like REINFORCE and/or REINFORCE with baseline (chapter 13 of the first cited work) or the very similar Vanilla Policy Gradient (VPG). We will then move on to Proximal Policy Optimization (PPO), Deep Deterministic Policy Gradient (DDPG)  and its evolutions Twin Delayed DDPG (TD3, seventh work cited) and DRQ (Data-regularized Q, fourth cited work).

As stated, in our benchmark we will consider both value-based methods and policy gradient methods. The selected algorithms are chosen to represent a wide range of approaches within both categories.

\subsubsection{Value-Based Methods}

Value-based methods are algorithms based on the approximation of a value function. The algorithms we will test in this category are:
\begin{itemize}
	\item \textbf{Deep Q-Network (DQN):} the first example of success in deep reinforcement learning, will serve as a sort of baseline for our benchmark.
	\item \textbf{RAINBOW~\cite{hessel:rainbow}:} an advanced method that combines several improvements to the original DQN, that will also be tested individually to assess their individual contributions to energy consumption and performance. These are listed hereafter:
	\begin{itemize}
		\item Double Q-Learning (Double DQN)~\cite{van:double_q};
		\item Prioritized Experience Replay~\cite{schaul:prioritized};
		\item Dueling Network Architectures~\cite{wang:dueling};
		\item Multi-step Learning~\cite{peng:incremental};
		\item Distributional RL~\cite{bellemare:distributional};
		\item Noisy Nets~\cite{fortunato:noisy};
	\end{itemize}
	\item \textbf{Self-Predictive Representations (SPR)~\cite{schwarzer:spr}:} a more advanced method introduced in recent research, which leverages self-predictive representations to improve efficiency.
\end{itemize}

\subsubsection{Policy Gradient Methods}

Policy gradient methods approximate the policy directly and include as a special case the actor-critic methods, which simultaneously approximate a policy and a value function. The algorithms we will test in this category are:
\begin{itemize}
	\item \textbf{REINFORCE~\cite[Chapter~13]{sutton:rl}:} a basic policy gradient method, or its variant REINFORCE with baseline (also known as Vanilla Policy Gradient, VPG).
	\item \textbf{Proximal Policy Optimization (PPO)~\cite{schulman:ppo}:} a popular and efficient policy gradient method that uses a clipped objective to improve training stability.
	\item \textbf{Deep Deterministic Policy Gradient (DDPG)~\cite{lillicrap:ddpg}:} an algorithm that combines policy gradients with deterministic policy updates for continuous action spaces.
	\item \textbf{Twin Delayed DDPG (TD3)~\cite{fujimoto:td3}:} an improvement over DDPG that addresses function approximation errors through various techniques, such as delayed policy updates and target policy smoothing.
	\item \textbf{Data-Regularized Q (DRQ)~\cite{kostrikov:drq}:} a method that incorporates data augmentation to regularize the training of Q functions, improving performance and stability.
\end{itemize}


\subsection{Task Selection}
\label{subsec:task_selection}

Regarding the task on which to compare the algorithms, there were several suitable candidates: Atari 100k~\cite{kaiser:atari100k}, one of the continuous control task of the DeepMind Control Suite, or one of the many other task (besides Atari) included in OpenAI Gymnasium (formerly Gym), and so on. After various tests and research we opted for the Atari 100k, a discrete task that consists of playing one of the Atari games for \num{100000} environment interactions.

The reason for this choice is multifaceted. Atari 100k is a widely used benchmark in the DRL community, the wealth of prior research and baseline results available facilitates a more straightforward validation and comparison of our experimental results with those from other studies and algorithms. It is well suited for evaluating the performance of almost all popular DRL algorithms, ensuring a comprehensive assessment. Additionally, Atari games provide a range of different challenges, including planning, reaction time, and strategy, making it a robust benchmark for assessing general DRL capabilities.

Moreover, the discrete nature of Atari 100k simplifies the implementation and comparison of algorithms, as continuous control tasks often require additional considerations and modifications. Finally, the \num{100000} interactions limit strikes a balance between providing enough data for meaningful evaluation and being computationally feasible within our resource constraints, especially considering the large number of experiments required for each algorithm, as detailed in section~\vref{subsubsec:number_runs}.

These factors combined make Atari 100k a practical and effective choice for our benchmark, enabling us to achieve our project goals efficiently.


\subsection{Experiment Setup}
\label{subsec:experiment_setup}
In this section we will address all the decisions made in the setup of the experiments.

\subsubsection{Number of Runs}
\label{subsubsec:number_runs}

In determining how many runs to carry out during the experimentation and testing of a reinforcement learning algorithm, at least two fundamental aspects must be taken into account: the high variance of reinforcement learning, and thus its high susceptibility to randomness, and the evaluation of the generality of the algorithm, which must therefore be tested in several different environments in order to actually prove that it is capable of solving multiple problems and not be ultra-specialized on a single use-case.

In addressing the first aspect we can refer to the literature to get an idea of how many runs with different seeds are usually used to alleviate this problem. If in the early days of RL (and not DRL) the number of runs stood at around 100 and in any case did not fall below 30 at least until the introduction of ALE (Arcade Learning Environment)~\cite{bellemare:ale} included, with the advent of DRL the number of runs was consistently reduced to 5 or less because of the high cost in terms of time and resources per run. Although this has been the standard for years, a more recent work~\cite{agarwal:statistical_precipice} has shown that using point estimates of aggregate performance such as mean and median scores across tasks is not the best way to summarize results because it ignores the statistical uncertainty inherent in using only a few runs.

In particular, the study points out that in the case of Atari at least 100 runs per environment are required to obtain robust results, a value that is, however, impractical in reality. In our case we will still be forced to limit ourselves to 4 runs per environment, but it should be considered how this is a less significant problem in our case than in other studies, since we are not attempting to advance the state of the art performance of DRL algorithms, but have instead a focus on energy consumption, which should in any case remain constant regardless of the actual learning of the agent, which is instead related to randomness.

Despite this, we will still attempt to use, in addition to the more classic and popular metrics such as the point estimates mentioned above, other metrics suggested in~\cite{agarwal:statistical_precipice}, designed precisely to obtain more efficient and robust estimates and have small uncertainty even with a handful of runs, not being overly affected by outliers.

With regard to the second aspect, namely, testing the algorithms on a variety of environments to evaluate their generality, Atari 100k once again comes to our aid, being constituted by 26 games. Moreover, the Arcade Learning Environment, built on top of the Atari 2600 emulator Stella and used by gymnasium, includes over 55 games. Unfortunately, again, we do not have the time and/or computational resources to test on all the Atari \num{100}k's 26 games or all the ones available in ALE, so we selected for the benchmark a representative subset of 8 Atari games, trying to choose games that cover a range of difficulties and styles. Obviously, with so few games because of the constraints just mentioned, an exhaustive selection is difficult, but we nonetheless tried to provide a balanced benchmark, ensuring that the selected games cover a range of challenges to effectively evaluate different algorithms, while still not being excessively difficult. This last requirement is due to basic DQN and its more simple extensions, which have some limitations in only 100k interactions (the team that introduced the DQNs trained its model on 2 million interactions to achieve interesting results).

Here are the 8 selected games, completed with a rationale for their inclusion:
\begin{itemize}
	\item \textit{Alien} - moderate difficulty, good for testing exploration and strategy.
	\item \textit{Amidar} - requires planning and quick decision-making.
	\item \textit{Assault} - yests reflexes and targeting accuracy.
	\item \textit{Boxing} - simple but requires precise control and timing.
	\item \textit{Breakout} - classic game, good for testing control.
	\item \textit{Freeway} - simple yet tests quick decision-making under pressure.
	\item \textit{Ms. Pac-Man} - classic maze game, tests navigation and evasion.
	\item \textit{Pong} - simple and well-understood, great for baseline comparisons.
\end{itemize}
%\begin{itemize}
%	\item \textit{Alien} - Moderate difficulty, good for testing exploration and strategy.
%	\item \textit{Amidar} - Requires planning and quick decision-making.
%	\item \textit{Assault} - Tests reflexes and targeting accuracy.
%	\item \textit{Asterix} - High-paced, tests navigation and reaction time.
%	\item \textit{Boxing} - Simple but requires precise control and timing.
%	\item \textit{Breakout} - Classic game, good for testing continuous control.
%	\item \textit{Chopper Command} - Involves shooting and evading, good for action-oriented strategies.
%	\item \textit{Demon Attack} - High intensity, tests reaction and aiming skills.
%	\item \textit{Freeway} - Simple yet tests quick decision-making under pressure.
%	\item \textit{Hero} - Complex game, requires planning and adaptability.
%	\item \textit{Krull} - Tests both navigation and strategic planning.
%	\item \textit{Ms. Pac-Man} - Classic maze game, tests navigation and evasion.
%	\item \textit{Pong} - Simple and well-understood, great for baseline comparisons.
%	\item \textit{Q*bert} - Tests navigation and planning in a constrained environment.
%	\item \textit{Seaquest} - Involves navigation, shooting, and resource management.
%	\item \textit{Montezuma's Revenge} - For testing exploration and sparse rewards.
%	\item \textit{Private Eye} - For long-term planning and memory.
%\end{itemize}
%resoconto finale di number of runs: in definitiva 8 giochi * 4 seed etc.
%poi dovrebbe venire come tracci ed analizzi i dati, sarebbe l'altra sezione praticamente, vedere se incorporarla qui o lasciarla lì, comunque weight and biases e code carbon
%infine ambiente di sviluppo e implementazione codice, quindi parlare di cleanrl.
%Given the high variance typical of reinforcement learning, it is essential to conduct multiple experiments with different random seeds to obtain robust results. Therefore, each algorithm will be evaluated on 15 different Atari games, with 4 runs per game using different random seeds. This approach ensures that our results are statistically significant and account for the inherent variability in RL training processes.

%in follow up magari si possono aggiungere altri giochi
The implementation of the experiments will be based on the cleanrl project, which provides a set of high-quality, standardized RL implementations. We will tweak the algorithms already implemented in cleanrl to match our use case and add the missing ones, adhering to the same philosophy and implementation principles to maintain consistency and comparability.

\begin{enumerate}
	\item \textbf{Environment Setup:} All algorithms will be implemented and run in a controlled environment to ensure consistent comparison.
	\item \textbf{Training Procedure:} Each algorithm will be trained for the same number of environment interactions to allow for a fair comparison of energy consumption and performance.
	\item \textbf{Energy Measurement:} The energy consumption of each algorithm will be measured using appropriate tools and methodologies.
	\item \textbf{Performance Evaluation:} The performance of each algorithm will be evaluated based on the scores achieved in the Atari 100k benchmark.
	\item \textbf{Multiple Runs:} Each algorithm will be run 4 times on 8 different games, using different random seeds for each run to ensure robustness of the results.
	\item \textbf{Additional Benchmark:} As an additional comparison, all algorithms will be re-trained until they reach the score obtained by the lowest performer in the initial benchmark, and their energy consumption and time to achieve this score will be recorded. This will be approximated using the returns from the logging of the training during the first trial.
\end{enumerate}

\subsection{Data Collection and Analysis}
\label{subsec:data_collection}

\begin{enumerate}
	\item Collect data on energy consumption and performance for each algorithm.
	\item Analyze the data to identify trade-offs between performance and energy consumption.
	\item Generate visualizations and statistical analyses to present the findings.
\end{enumerate}
